\documentclass{article}

% Language setting
% Replace `english' with e.g. `spanish' to change the document language
\usepackage[italian]{babel}
\usepackage{float}
\usepackage{subfig}

% Set page size and margins
% Replace `letterpaper' with `a4paper' for UK/EU standard size
\usepackage[letterpaper,top=2cm,bottom=2cm,left=3cm,right=3cm,marginparwidth=1.75cm]{geometry}

% Useful packages
\usepackage{amsmath}
\usepackage{graphicx}
\usepackage{float}
\usepackage[colorlinks=true, allcolors=blue]{hyperref}

\setlength{\parskip}{6pt}%

\title{Simulatore del movimento dei pianeti in un sistema solare}
\author{Gioele Mancino, Federico Mastroforti, Luca Tesei, Nico Tortorici}

\begin{document}
\maketitle
\section{Il Progetto}
\subsection{Istruzioni per la compilazione}
All'interno del compilatore è necessario scaricare la repository online utile per la creazione della GUI:
\begin{verbatim}
   $ sudo add-apt-repository ppa:texus/tgui
   $ sudo apt-get update
   $ sudo apt-get install libtgui-1.0-dev
    \end{verbatim}
Poi possiamo compilare tramite CMake usando questi comandi:
\begin{verbatim}
    $ cmake -S. -B build
    $ cmake --build build
    $ build/gravity
     \end{verbatim} 

\subsection{Obiettivo del simulatore}
Si vuole creare un programma in grado di calcolare in tempo reale e rappresentare tramite
un'interfaccia il movimento di alcuni corpi celesti e le interazioni tra loro. è inoltre possibile aggiungere nuovi pianeti direttamente all'interno della simulazione che interagiranno con quelli già presenti.

\subsection{User interaction}
\section{Strumenti di sviluppo}
\subsection{CMake}
\subsection{ClangFormat}
\subsection{Git e Github}
\section{Strategie utilizzate}

\end{document}