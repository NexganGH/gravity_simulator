\documentclass{article}

% Language setting
% Replace `english' with e.g. `spanish' to change the document language
\usepackage[italian]{babel}
\usepackage{float}
\usepackage{subfig}

% Set page size and margins
% Replace `letterpaper' with `a4paper' for UK/EU standard size
\usepackage[letterpaper,top=2cm,bottom=2cm,left=3cm,right=3cm,marginparwidth=1.75cm]{geometry}

% Useful packages
\usepackage{amsmath}
\usepackage{graphicx}
\usepackage{float}
\usepackage[colorlinks=true, allcolors=blue]{hyperref}

\setlength{\parskip}{6pt}%

\title{Simulatore del movimento dei pianeti in un sistema solare}
\author{Gioele Mancino, Federico Mastroforti, Luca Tesei, Nico Tortorici}

\begin{document}
\maketitle
\section{Il Progetto}
\subsection{Istruzioni per la compilazione}
All'interno del compilatore è necessario scaricare la repository online utile per la creazione della GUI:
\begin{verbatim}
   $ sudo add-apt-repository ppa:texus/tgui
   $ sudo apt-get update
   $ sudo apt-get install libtgui-1.0-dev
    \end{verbatim}
Poi possiamo compilare tramite CMake usando questi comandi:
\begin{verbatim}
    $ cmake -S. -B build
    $ cmake --build build
    $ build/gravity
     \end{verbatim} 

\subsection{Obiettivo del simulatore}
Si vuole creare un programma in grado di calcolare in tempo reale e rappresentare tramite
un'interfaccia il movimento di alcuni corpi celesti e le interazioni tra loro. è inoltre possibile aggiungere nuovi pianeti direttamente all'interno della simulazione che interagiranno con quelli già presenti.

\subsection{User interaction}
Nell'angolo alto a sinistra dell'interfaccia grafica sono presenti due tasti: \\
-il tasto Play/Pause (Figura \href{playpause}, 1 ) che permette di far partire o fermare la simulazione;\\
-il tasto Reset (Figura \href{playpause}, 2) che riporta tutto allo stato iniziale (i corpi già presenti ritornano nel punto in cui erano a inizio simulazione mentre quelli generati dall'utente vengono eliminati); \\
In tutta la finestra grafica è possibile cliccare con il tasto destro in un punto, facendo ciò comparirà una finestra in cui è possibile inserire la massa del corpo che stiamo inserendo (in unità di massa terrestre) e sarà poi possibile selezionare il verso di movimento iniziale del corpo

\begin{figure} [H]
    \centering
    \includegraphics[height=.42\linewidth]{Playpause.png}
    \captionof{figure}{Il tasto (1) Play/Pause permette di far partire e fermare la simulazione in ogni momento}
  \label{inclinometro}
\end{figure}

\section{Strumenti di sviluppo}
\subsection{C++}
\subsection{Librerie esterne}
\subsection{CMake}
\subsection{VSCode}
\subsection{Git e Github}
\section{Implementazione}
\subsection{Metodo Leapfrog}
\subsection{Struttura del progetto}
\section{Testing e debugging}
\section{Risultati e problemi riscontrati}
\appendix 
\section{Calcoli e approssimazioni}

\end{document}